\section{Related works}\label{related_work}

Music genre classification (MGC) is at this point a well-known research problem and a subdomain of Music Information Retrieval (MIR). Culture and therefore music avoids strict barriers and definitions, nevertheless, each piece of music is usually categorized into one or more genres. MGC enables us to study this categorization, explore similarities and differences between various genres or even construct a taxonomy. 

In the past, due to heavy computational limitations, the main focus of MGC was put on finding the best features for classification purposes. In \cite{oldFeatures} such features were e.g. \textit{AverageSyllablesPerWord} or \textit{SentenceLengthAverage}. Naturally, word embedding played an important role in extracting information from lyrics and the use of simple methods like \textit{bag-of-words} can be found in various papers \cite{mgc_example_1, liang2011music}. With time, an increasing amount of focus was put strictly on embeddings themselves, developing novel and improved representations. 

Currently, all state-of-the-art approaches for MGC utilizing lyrics rely heavily on word embeddings. In a recent publication \cite{musicWordEmbed} an attempt was made to train the embedding model strictly on lyrics. Unfortunately, the significance of the work is hard to assess due to the lack of usage of this model.

It is also rather common to approach MGC in a multi-modal manner. Usage of the audio itself has to be second if not the most popular source of information with many published articles \cite{audio_1dcnn, audio_attention, audio_reviews_cover, oldFeatures, oldAudio}. Other less trivial data sources are symbolic \cite{symbolic}, culture \cite{oldFeatures}, text reviews \cite{audio_reviews_cover}, and cover art \cite{audio_reviews_cover}. One could say that at this stage researchers experiment with enriching the pieces of music with any meaningful data possible.

The datasets that we work on are:
\begin{itemize}
    \item \textit{Song lyrics from 79 musical genres} dataset from Kaggle website \cite{KaggleDataset},
    \item \textit{MetroLyrics} dataset processed and put in a GitHub repository \cite{GithubDataset},
    \item our own dataset created using Spotify API \cite{Spotify} and Genius API \cite{Genius}.
\end{itemize}

