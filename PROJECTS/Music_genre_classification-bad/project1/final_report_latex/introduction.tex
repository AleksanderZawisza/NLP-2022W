\section{Introduction}
A music genre is a conventional label on the musical piece which characterizes it as having certain features, conventions, or characteristics. It is quite a complicated problem to say precisely how genres are distinguished. The genre often dictates the style and rhythm of the audio of the song. It seems much harder to define the music genre by lyrics alone, even from a human perspective. Therefore, it is quite an interesting topic to try making such a distinction based on song text. Similar research has already been conducted, but this topic is yet to be fully explored.

The song's lyrics are often related to its melody and rhythm. It is also common for different genres to raise different topics. It was already shown that a combination of audio and text features gets better results than using only audio features \cite{mayer2011Ref}. Furthermore, lyrics may be more accessible and easier to process than audio. Therefore, lyrics classification seems to be an interesting field of study both for its own and for its potential connection with audio features.

In this research, we explore different methods for lyrics-based genre classification. Our study includes testing different methods of obtaining text embeddings, such as GloVe, word2vec, BERT, and varying classification models, such as Naive Bayes, Linear Support Vector Machine, XGBoost, and Convolutional Neural Network. This kind of analysis is currently lacking in the field of MGC.

We also test if adding additional information in the form of the title of the song may improve the accuracy of the model. This is interesting to see as in many songs the title is often repeated in the song lyrics. That means that the title may be actually redundant.

The research paper is divided into multiple sections. In section \ref{related_work} we describe related works in the domain of MGC. Section \ref{approach} presents used datasets and the preprocessing done. Futhermore, used methods and models are characterized. In section \ref{experiments} we demonstrate exploratory data analysis, conducted experiments and obtained results, which are discussed in section \ref{discussion}. The whole project is concluded in section \ref{conclusion} and the future work that can be done in this subject is described in section \ref{future_work}.
