\section{Introduction}
% Skąd się biorą te odstępy????
A music genre is a conventional label on the musical piece which characterizes it as having certain features, conventions, or characteristics. It is quite a complicated problem to say precisely how genres are distinguished. The genre often dictates the style and rhythm of the audio of the song. It seems much harder to define the music genre by lyrics alone, even from a human perspective. Therefore, it is quite an interesting topic to try making such a distinction based on song text. Similar research has already been conducted, but this topic is yet to be fully explored.

The song's lyrics are often related to its melody and rhythm. It is also common for different genres to raise different topics. It was already shown that a combination of audio and text features gets better results than using only audio features \cite{mayer2011Ref}. Furthermore, lyrics may be more accessible and easier to process than audio. Therefore, lyrics classification seems to be an interesting field of study both for its own and for its potential connection with audio features.

In this research, we want to explore different methods for lyrics-based genre classification. Our study will include testing different methods of obtaining text embeddings, such as Continuous Bag-
of-Words, GloVe, word2vec, BERT, and varying classification models, such as Naive Bayes, Support Vector Machine, XGBoost, and Convolutional Neural Network.

We also want to include in our research sentiment analysis of the text. One of the characteristics of the music genre, though rarer considered, is the emotion that the song conveys. We decided to check how exactly those two relate since there seems to be sparse similar research. Therefore, for the above genre classification task, we will additionally consider the emotion detected in the song lyrics and with the use of fusion techniques check how it influences classification performance.
